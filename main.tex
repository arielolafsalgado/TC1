\documentclass{article}
\usepackage[utf8]{inputenc}
\usepackage[legalpaper, portrait, margin=0.5in]{geometry}
\usepackage[spanish,es-tabla]{babel}

\title{Trabajo Práctico 1: Conceptos básicos}
\author{Emmanuel Ferreyra, Bruno Kaufman, Ariel Salgado}
\date{\today}

\usepackage{natbib}
\usepackage{graphicx}

\begin{document}

\maketitle

\section{Problema 1}
En este punto se pide analizar tres redes complejas de interacción de proteinas. La primera se trata de interacciones binarias entre proteinas, la segunda de co-pertenencia a complejos, y la tercera una obtenida del proyecto Yeast Interactome Database (YID).

\subsection{Punto 1.a}
En este inciso se solicita visualizar las redes de forma informativa acerca de su estructura. Las tres redes se visualizan en la tabla \ref{comparacionvisualP1}.
%% Fijate de incluir una mención a cada red en el texto, que no esté solo en la figura. Y también poner las redes "completas", no cortadas en dos.
\begin{table}[ht]
\caption{\textbf{Superior:} Componente gigante perteneciente a la red del proyecto YID (Y2H), de interacciones binarias (APMS), y de co-pertenencia (LIT), de izquierda a derecha. \textbf{Inferior:} misma información respecto a las componentes pequeñas de las redes.}
\centering
\begin{tabular}{ccc}
\includegraphics[scale=0.4]{Imagenes_P1/Y2Hgigante.pdf}&\includegraphics[scale=0.4]{Imagenes_P1/APMSgigante.pdf}&\includegraphics[scale=0.4]{Imagenes_P1/LITgigante.pdf}\\
 
\includegraphics[scale=0.4]{Imagenes_P1/Y2Hchiquitos.pdf}&\includegraphics[scale=0.4]{Imagenes_P1/APMSchiquitos.pdf}&\includegraphics[scale=0.4]{Imagenes_P1/LITchiquitos.pdf}\\
\end{tabular}
\label{comparacionvisualP1}
\end{table}

Se pueden observar algunas cosas en base a estos diagramas. En primer lugar, observando sólamente a los componentes pequeños podemos ver que en la red Y2H, no presentan clustering considerable. En contraste, la red APMS presenta clustering significativo, y la red LIT también, aunque no tanto como la red APMS.

Por otro lado, observando el componente gigante se puede ver que la mayoría de los links de la red Y2H tienden hacia un mismo lugar en la red, mientras que la red APMS se presenta menos centralizada y la LIT toma un punto medio entre las dos.

\subsection{Punto 1.b}
En este inciso se pide resumir ciertas características de red en una tabla. La siguiente se puede ver en la tabla \ref{tablacantidadesredesP1}.

\begin{table}[ht]
\caption{Cantidades de red asociadas a las redes correspondientes.}
\centering
\resizebox{\textwidth}{!}{%
\begin{tabular}{lccc}
\textbf{Red} & \textbf{Y2H} & \textbf{APMS} & \textbf{LIT}\\
Número de nodos & 2018 & 1622 & 1536\\
Número de ejes & 2930 & 9070 & 2925\\
Grado medio & 2.904 & 11.184 & 3.809\\
Densidad & 1.440 10\textsuperscript{-3} & 6.899 10\textsuperscript{-3} & 2.481 10\textsuperscript{-3}\\
Clustering global & 2.361 10\textsuperscript{-2} & 6.186 10\textsuperscript{-1} & 3.462 10\textsuperscript{-1}\\
Clustering local medio & 9.700 10\textsuperscript{-2} & 7.410 10\textsuperscript{-1} & 4.556 10\textsuperscript{-1}\\
Diámetro & 14 & 15 & 19\\

\end{tabular}}
\label{tablacantidadesredesP1}
\end{table}

Se puede ver que efectivamente la red Y2H presenta un clustring significativamente menos a las otras dos, y que la APMS tiene uno mayor a la red LIT, aunque están en el mismo órden de magnitud.

Además, es interesante ver que el diámetro nos informa que para llegar de un extremo al otro de la red LIT, el número máximo es apreciablemente mayor que en las otras dos. Esto no es fácilmente visible en los diagramas vistos en la tabla \ref{comparacionvisualP1}

\subsection{Punto 1.c}

Las interacciones reportadas son todas de naturaleza biomolecular. Por un lado, es razonable que exista una sola componente gigante para estas redes, ya que las redes biológicas presentan una distribución de grado libre de escala, consistente con un mecanismo de attachment preferencial: en otras palabras, cuanto mayor el grado de un nodo, mayor la probabilidad de que tenga un dado enlace con otro. Entonces, la componente más grande siempre atraerá la mayoría de los enlaces, y se volverá más grande, llegando a ser la única componente gigante.

Por otro lado, es razonable que una red de co-pertenencia (LIT) tenga una clusterización significativa, ya que muchas proteínas pueden pertenecer a un mismo complejo, y así todas estarán conectadas entre si. Suponiendo que esta co-pertenencia influye sobre las interacciones binarias, tiene también sentido que una clusterización alta se presente en la red de interacciones binarias (APMS).

Además, la red de copertenencia (LIT) presenta un diámetro más grande que las otras dos. Esto es razonable, porque cuando se trata de interacciones binarias, es posible que interactúen dos proteínas que no estén en el mismo complejo, agregando conexiones y disminuyendo el diámetro.

Esto también explica el grado medio mayor de la red de interacciones binarias (APMS), pero no la del interactoma de la levadura (Y2H), que presenta el grado medio más pequeño de todas. Esto puede ser por falsos negativos: al observarse sólamente aquello que ocurre en el núcleo, van a haber muchas interacciones que se pierden. Esto también explicaría el clustering relativamente pequeño del interactoma de la levadura.

\section{Problema 2}
En el segundo problema se propone analizar una red social de 62 delfines de Nueva Zelanda. La red consta de 34 machos, 24 hembras y 4 delfines de sexo desconocido. La base de datos establece 159 vinculos entre los delfines de la red.

\subsection{Punto 2.a.}
El primer objetivo es comparar diferentes opciones de layout para graficar la red, comparando ventajas y desventajas. Para esto empleamos el paquete \texttt{igraph} de \texttt{R}. El mismo provee distintas opciones de layout a partir de funciones predeterminadas. Presentamos para ejemplificar tres layouts: \texttt{layout-with-fr},\texttt{layout-as-tree} y \texttt{layout-on-grid}. El primero construye el layout a partir de un cálculo de equilibrio de fuerzas, el segundo intenta representar al grafo como un árbol, permitiendo que se le indique un nodo semilla, y el tercero posiciona los nodos sobre una grilla regular. En la figuras \ref{pt2layouts} se pueden observar los mismos. Los nodos rojos representan los machos, los azules las hembras, y los negros los de sexo desconocido. Ninguno de los últimos dos facilita la visualización del gráfico, indicando que las estructuras propuestas no son representativas de la estructura del grafo. Por otro lado, \texttt{layout\_with\_fr}, al tener directamente en cuenta los vinculos de la red para su construcción, nos permite observar que hay dos grupos de delfines bastante diferenciados, uno en el que predominan las hembras (y dominan los colores azules) y otro donde predominan los machos (mayoría de colores rojos).

\begin{figure}[!htb]
   \begin{minipage}{0.3\textwidth}
	\centering
	\includegraphics[width=.7\linewidth]{Imagenes_P1/layout_dolphins2.pdf}
	\caption{Red de los delfines, graficada según \textt{layout\_with\_fr}.}
	\label{pt2layoutsfr}
   \end{minipage}\hfill
   \begin{minipage}{0.3\textwidth}
	\centering
	\includegraphics[width=.7\linewidth]{Imagenes_P1/layout_dolphins3.pdf}
	\caption{Red de los delfines, graficada según \textt{layout\_as\_tree}.}
	\label{pt2layouttree}
   \end{minipage}\hfill
   \begin{minipage}{0.3\textwidth}
	\centering
	\includegraphics[width=.7\linewidth]{Imagenes_P1/layout_dolphins7.pdf}
	\caption{Red de los delfines, graficada según \textt{layout\_on\_grid}.}
	\label{pt2layoutgrid}
   \end{minipage}
   \label{pt2layouts}
\end{figure}


\subsection{Punto 2.b.}
En este punto se propone testear la hipótesis que sugiere \texttt{layout\_with\_fr}: que hay más vinculos entre delfines del mismo sexo (vinculos homofilos) que entre delfines de distinto sexo (vinculos heter\'ofilos). Para esto, calculamos la fracción de ejes que conectan delfines de distinto sexo como

\begin{equation}
    f_{mf} = \frac{1}{M}\sum_{ij} A_{ij} \delta_{s_i,m} \delta_{s_j,f}
\end{equation}

donde $s_i$ es el sexo del delfin $i$ y $A_{ij}$ es la matriz de adyacencia del red, y $M = \sum_{ij} A_{ij}/2$ es la cantidad de ejes en la red. De forma análoga, calculamos las fracciones de ejes que conectan hembras con hembras y machos con machos:

\begin{equation}
        f_{m} = \frac{1}{2M}\sum_{ij} A_{ij} \delta_{s_i,m} \delta_{s_j,m} 
\end{equation}
\begin{equation}
        f_{f} = \frac{1}{2M}\sum_{ij} A_{ij} \delta_{s_i,f} \delta_{s_j,f} 
\end{equation}
En todos los casos, consideramos únicamente los defines con sexo conocido. Obtenemos los valores de $f_{mf} = 0.327$, $f_{m} = 0.377$ y $f_{f} = 0.226$, quedando aproximadamente un $7\%$ de los ejes sin considerar por no conocerse el sexo. Para saber si estos valores son altos o bajos, el ejercicio propone reasignar de forma azarosa los sexos de cada delfín, y calcular las tres fracciones para cada asignación de sexo. Realizando 1000 reasignaciones, obtenemos tres distribuciones de valores. Podemos observar estos en la figura \ref{pt2histogramas}. Obtenemos valores esperados de $\widehat{f_{mf}} = 0.43 \pm 0.04$, $\widehat{f_{m}} = 0.30 \pm 0.04$ y $\widehat{f_{f}} = 0.15 \pm 0.03$, de donde vemos que los valores observados se encuentran entre 1.8 y 2.8 sigmas de distancia de los esperados. Si calculamos los $p$ valores en cada caso, obtenemos $p_{mf} = 0.005$,$p_m = 0.029$ y $p_f = 0.015$, indicando una muy baja probabilidad de que por azar la estructura observada se forme.

\begin{figure}[!htb]
   \begin{minipage}{0.3\textwidth}
	\centering
	\includegraphics[width=.7\linewidth]{Imagenes_P1/histo_homofilia1.pdf}
	\caption{Histograma de fracción de ejes que conectan delfines de sexos distintos.}
	\label{pt2histo-cruces}
   \end{minipage}\hfill
   \begin{minipage}{0.3\textwidth}
	\centering
	\includegraphics[width=.7\linewidth]{Imagenes_P1/histo_homofilia2.pdf}
	\caption{Histograma de fracción de ejes que conectan defines macho con macho.}
	\label{pt2histo-mym}
   \end{minipage}\hfill
   \begin{minipage}{0.3\textwidth}
	\centering
	\includegraphics[width=.7\linewidth]{Imagenes_P1/histo_homofilia3.pdf}
	\caption{Histograma de fracción de ejes que conectan defines hembra con hembra.}
	\label{pt2histo-fyf}
   \end{minipage}
   \label{pt2layout}
\end{figure}

\subsection{Punto 2.c.}
Por último, el ejercicio propone comparar distintas estrategias para remover nodos de la red, basandose en la estructura de la misma. Para esto proponemos 4 estrategias: extraer nodos según su grado, su coeficiente local de clustering, su betwenness (una medida de la cantidad de caminos mínimos que atraviezan un eje) y extraer completamente al azar. En la figura \ref{rupturas} podemos observar el tamaño de la primer y segunda componente. Elegimos la estrategia más apropiada viendo cual asemeja los tamaños más prontamente. Vemos que la estrategia más exitosa es la que extrae por betweenness. Seguida de ella está extraer por grado, que tiene éxito cuando las componentes ya son más pequeñas. Por último, extraer según el coeficiente de clustering no da un resultado muy distinto del azar.
\bibliographystyle{plain}
\bibliography{references}
\end{document}
